% +++
% latex = "uplatex"  # or "platex"
% +++
\documentclass[autodetect-engine, dvi=dvipdfmx]{wtarticle}

\setdocinfo{
  title = {サンプル文書},
  author = {wtsnjp},
}

\setdocstyle{
  packages = {
    newtxtext, newtxmath, bxjalipsum
  },
  emph-style = \sffamily\gtfamily,
}

\begin{document}

\maketitle

これは,フォント設定確認用の\emph{サンプル文書 (sample document) }です.

\section{基本書体}

まずは本文で頻繁に使用するものを確認しましょう:
%
\begin{itemize}
\item メイン(明朝)のフォント.英数字としてABCや123もチェック.
\item \emph{強調(ゴシック)のフォント.英数字としてABCや123もチェック.}
\end{itemize}

\section{低レイヤ}

念のため,低レイヤの設定状況も確認しておきます:
%
\begin{itemize}
\item \textmc{「明朝体」のフォント.\textbf{その「ボールド」.}}
\item \textgt{「ゴシック体」のフォント.\textbf{その「ボールド」.}}
\end{itemize}

\section{長文から受ける印象}

\jalipsum[1-2]{wagahai}%
\emph{\jalipsum[3]{wagahai}}%
\jalipsum[4-5]{wagahai}

\end{document}
